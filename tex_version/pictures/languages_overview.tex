% !TeX document-id = {543bad56-0a75-4f2c-a11d-a94f42751e3d}
% !TeX TXS-program:compile = txs:///xelatex/[--shell-escape]
\documentclass[border=20pt,varwidth=\maxdimen]{standalone}
\renewcommand\familydefault{\sfdefault} % Default family: serif 
\usepackage[usenames,dvipsnames]{xcolor}
\usepackage{tikz}
\usetikzlibrary{calc} 
\usetikzlibrary{arrows, decorations.markings,positioning,backgrounds,shapes}
\usetikzlibrary{patterns}
\usetikzlibrary{fit}
\usepackage[normalem]{ulem}
\usepackage{minted}

% Colors for c, assembly and machine codes.
\definecolor{c}{HTML}{002f55}
\definecolor{a}{HTML}{A5ACAF}
\definecolor{m}{HTML}{00AEEF}
% Colors for compiler and assembler.
\definecolor{compiler}{HTML}{64A0C8}
\definecolor{assembler}{HTML}{44D62C}

\begin{document}
% C Code
\newsavebox{\ccode}
\begin{lrbox}{\ccode}
	\RecustomVerbatimEnvironment{Verbatim}{BVerbatim}{}%
	\begin{minted}{c}
printf("Hello, ");
printf("World!");
\end{minted}
\end{lrbox}
% Python Code 
\newsavebox{\pycode}
\begin{lrbox}{\pycode}
	\RecustomVerbatimEnvironment{Verbatim}{BVerbatim}{}%
	\begin{minted}{Python}
print("Hello, ");
print("World!");
	\end{minted}
\end{lrbox}
% Python Code 
\newsavebox{\cscode}
\begin{lrbox}{\cscode}
	\RecustomVerbatimEnvironment{Verbatim}{BVerbatim}{}%
	\begin{minted}{csharp}
Console.Write("Hello, ");
Console.Write("World!");
\end{minted}
\end{lrbox}
% IL code
% Taken from https://www.developer.com/microsoft/c-sharp/c-and-intermediate-language-il/
\newsavebox{\ilcode}
\begin{lrbox}{\ilcode}
	\RecustomVerbatimEnvironment{Verbatim}{BVerbatim}{}%
	\begin{minted}{text}
.maxstack  8
IL_0000:  ldstr "Hello"
IL_0005:  call  void [mscorlib]
System.Console::WriteLine(string)
\end{minted}
\end{lrbox}

% Machine code
% Extracted from what you get if you run xxd -b a.out after compiling the following source code with gcc:
%int main(void)
%{
%	int age = 10;
%	char initial = 'C';
%}
\newsavebox{\mcode}
\begin{lrbox}{\mcode}
	\RecustomVerbatimEnvironment{Verbatim}{BVerbatim}{}%
	\begin{minted}{text}
01000010 01001001
00000000 00101110
	\end{minted}
\end{lrbox}

\newsavebox{\mcodea}
\begin{lrbox}{\mcodea}
	\RecustomVerbatimEnvironment{Verbatim}{BVerbatim}{}%
	\begin{minted}{text}
01000010 01001001
	\end{minted}
\end{lrbox}

\newsavebox{\mcodeb}
\begin{lrbox}{\mcodeb}
	\RecustomVerbatimEnvironment{Verbatim}{BVerbatim}{}%
	\begin{minted}{text}
01000010 01001001
	\end{minted}
\end{lrbox}

\begin{tikzpicture}[
		block/.style={
				rectangle,
				thick,
				minimum width=5em,
				align=center,
				rounded corners,
				minimum height=2em
			},
		data/.style={
				cylinder,
				draw=black,
				thick,
				aspect=0.7,
				minimum height=1.7cm,
				minimum width=1.5cm,
				shape border rotate=90,
				cylinder uses custom fill,
				cylinder body fill=red!30,
				cylinder end fill=red!10
			}
	]
	% Code blocks
	%% High-level Language
	\node [color=c] (c) {High-Level Language};
	%%% Compiled
	\node [below left = .3cm and .1cm of c, block, draw=c] (compiled) {Compiled};
	\node [below right = -.5cm and .2cm of compiled, block, fill=c!20, draw=c] (compiledcode) {\usebox\ccode};
	%%% Interpreted
	\node [below  = 1cm of compiled, block, draw=c] (interpreted) {Interpreted};
	\node [below right = -.5cm and .2cm of interpreted, block, fill=c!20, draw=c] (interpretedcode) {\usebox\pycode};
	%%% Managed
	\node [below = 2cm of interpreted, block, draw=c] (managed) {Managed};
	\node [below right = -.5cm and .2cm of managed, block, fill=c!20, draw=c] (managedcode) {\usebox\cscode};
	%% Assembly
	\node [right = 5cm of c, color=a] (assemb) {Intermediate Language};
	\node [below = 4.5cm of assemb, block, fill=a!20, draw=a] (ilcode) {\usebox\ilcode};
	%% Machine-level Language
	\node [right = 5cm of assemb, color=m] (machine) {Machine Language};
	\node [below = .5cm of machine, block, fill=m!20, draw=m] (machinecode1) {\usebox\mcode};
	\node [below = 2cm of machine, block, fill=m!20, draw=m] (machinecode2a) {\usebox\mcodea};
	\node [below = 3cm of machine, block, fill=m!20, draw=m] (machinecode2b) {\usebox\mcodeb};
	\node [below = 4.5cm of machine, block, fill=m!20, draw=m] (machinecode3a) {\usebox\mcodea};
	\node [below = 5.5cm of machine, block, fill=m!20, draw=m] (machinecode3b) {\usebox\mcodeb};
	%% Output
	\node [right = 3cm of machine] (output) {Output (on screen)};
	\node [below = .5cm of output, block, draw=black] (output1) {Hello, World!};
	\node [below = 2cm of output, block, draw=black] (output2a) {Hello,};
	\node [below right = 3cm and -1.7cm of output, block, draw=black] (output2b) {\phantom{Hello, }World!};
	\node [below = 4.5cm of output, block, draw=black] (output3a) {Hello,};
	\node [below right = 5.5cm and -1.7cm of output, block, draw=black] (output3b) {\phantom{Hello, }World!};
	% Arrows
	%% Compiler
	\draw[thick, ->, color=compiler] (compiledcode)  to node[above]{Compiler} (machinecode1) ;
	\draw[thick, ->, color=compiler] (managedcode)  to node[above]{Compiler} (ilcode) ;
	%% Interpreter
	\draw[thick, ->, color=assembler] ($(interpretedcode.east)+(0, .35)$)  to node[above]{Interpreter} (machinecode2a) ;
	\draw[thick, ->, color=assembler] ($(interpretedcode.east)+(0, -.35)$)  to node[above]{Interpreter} (machinecode2b) ;
	\draw[thick, ->, color=assembler] ($(ilcode.east)+(0, .55)$)  to node[above]{Interpreter} (machinecode3a) ;
	\draw[thick, ->, color=assembler] ($(ilcode.east)+(0, -.5)$)  to node[above]{Interpreter} (machinecode3b) ;
	% Execution
	\draw[thick, ->] (machinecode1)  to node[above]{Execution} (output1);
	\draw[thick, ->] (machinecode2a)  to node[above]{Execution} (output2a);
	\draw[thick, ->] (machinecode2b)  to node[above]{Execution} (output2b);
	\draw[thick, ->] (machinecode3a)  to node[above]{Execution} (output3a);
	\draw[thick, ->] (machinecode3b)  to node[above]{Execution} (output3b);
\end{tikzpicture}

\end{document}